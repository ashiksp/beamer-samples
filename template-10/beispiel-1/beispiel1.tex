% Dieser Text ist urheberrechtlich gesch�tzt
% Er stellt einen Auszug eines von mir erstellten Referates dar
% und darf nicht gewerblich genutzt werden
% die private bzw. Studiums bezogen Nutzung ist frei
% Nov. 2004
% Autor: Sascha Frank 
% Universit�t Freiburg 
% www.informatik.uni-freiburg.de/~frank/

\documentclass{beamer}

\usepackage{beamerthemesplit}
\usepackage{graphics}

\begin{document}

\title{Beispiel 1}
\author{Miriam Seel \& Sascha Frank}

\section{Kleiner Test}

\frame{\titlepage}

\frame{Kleiner Test}
\frame{
\transduration<1>{0}
\invisible<2>{\includegraphics[scale=0.5]{bild1}}
\invisible<1>{Wieviele Kugeln waren es ?}
}
\frame{
\includegraphics[scale=0.5]{bild1}}

\frame{}
\frame{
\transduration<1>{0}
\invisible<2>{\includegraphics[scale=0.5]{bild2}}

\invisible<1>{Wieviele Kugeln waren es diesmal ?}
}
\frame{
\includegraphics[scale=0.5]{bild2}}
\frame{}
\frame{
\transduration<1>{0}
\invisible<2>{\includegraphics[scale=0.5]{bild3}}
\\
\invisible<1>{}
\invisible<1>{Wieviele Kugeln waren es jetzt ?}
\invisible<1,2>{Vorteil der B\"undelung}
}
\frame{
\includegraphics[scale=0.5]{bild3}}
 

\end{document}